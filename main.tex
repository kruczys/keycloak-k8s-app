\documentclass[12pt,a4paper]{article}
\usepackage[utf8]{inputenc}
\usepackage[polish]{babel}
\usepackage{dsfont} 
\usepackage{amsmath}
\usepackage{graphicx}
\usepackage[top=1in, bottom=1.5in, left=1.25in, right=1.25in]{geometry}
\usepackage{subfig}
\usepackage{multirow}
\usepackage{multicol}
\graphicspath{{Imagens/}}
\usepackage{xcolor,colortbl}
\usepackage{float}
\usepackage{fancyhdr} % Required for custom headers
\usepackage{lastpage} % Required to determine the last page for the footer
\usepackage{extramarks} % Required for headers and footers
\usepackage{indentfirst}
\usepackage{placeins}
\usepackage{scalefnt}
\usepackage{xcolor,listings}
\usepackage{textcomp}
\usepackage{color}
\usepackage{verbatim}
\usepackage{framed}
\usepackage{hyperref} % Optional, for clickable links

\definecolor{codegreen}{rgb}{0,0.6,0}
\definecolor{codegray}{rgb}{0.5,0.5,0.5}
\definecolor{codepurple}{HTML}{C42043}
\definecolor{backcolour}{HTML}{F2F2F2}
\definecolor{bookColor}{cmyk}{0,0,0,0.90}  
\color{bookColor}

\lstset{upquote=true}

\lstdefinestyle{mystyle}{
	backgroundcolor=\color{backcolour},   
	commentstyle=\color{codegreen},
	keywordstyle=\color{codepurple},
	numberstyle=\numberstyle,
	stringstyle=\color{codepurple},
	basicstyle=\footnotesize\ttfamily,
	breakatwhitespace=false,
	breaklines=true,
	captionpos=b,
	keepspaces=true,
	numbers=left,
	numbersep=10pt,
	showspaces=false,
	showstringspaces=false,
	showtabs=false,
}
\lstset{style=mystyle}

\newcommand\numberstyle[1]{%
	\footnotesize
	\color{codegray}%
	\ttfamily
	\ifnum#1<10 0\fi#1 |%
}

\definecolor{shadecolor}{HTML}{F2F2F2}

\newenvironment{sqltable}%
{\snugshade\verbatim}%
{\endverbatim\endsnugshade}

% Margins
\addtolength{\footskip}{0cm}
\addtolength{\textwidth}{1.4cm}
\addtolength{\oddsidemargin}{-.7cm}
\addtolength{\textheight}{1.6cm}

% Paragraph spacing
\addtolength{\parskip}{.2cm}

% Header and footer setup
\pagestyle{fancy}
\rhead{\hmwkAuthorName} % Top left header
\lhead{\hmwkClass: \hmwkTitle} % Top center header
\rhead{\firstxmark} % Top right header
\lfoot{Konrad Kreczko} % Bottom left footer
\cfoot{} % Bottom center footer
\rfoot{} % Bottom right footer
\renewcommand{\headrulewidth}{1pt}
\renewcommand{\footrulewidth}{1pt}

\newcommand{\hmwkTitle}{Aplikacja do recenzji filmów} % Tytuł projektu
\newcommand{\hmwkDueDate}{\today} % Data 
\newcommand{\hmwkClass}{Technologie chmurowe} % Nazwa przedmiotu
\newcommand{\hmwkAuthorName}{Konrad Kreczko} % Imię i nazwisko

\begin{document}

% Strona tytułowa
\begin{titlepage}
    \vfill
	\begin{center}
	\hspace*{-1cm}
	\vspace*{0.5cm}
    \includegraphics[scale=0.55]{imagens/loga.png}\\
	\textbf{Uniwersytet Gdański \\ [0.05cm]Wydział Matematyki, Fizyki i Informatyki \\ [0.05cm] Instytut Informatyki}

	\vspace{0.6cm}
	\vspace{4cm}
	{\huge \textbf{\hmwkTitle}}\vspace{8mm}
	
	{\large \textbf{\hmwkAuthorName}}\\[3cm]
	
	\hspace{.45\textwidth} %posiciona a minipage
	   \begin{minipage}{.5\textwidth}
	   Projekt z przedmiotu technologie chmurowe na kierunku informatyka profil praktyczny na Uniwersytecie Gdańskim.\\[0.1cm]
	  \end{minipage}
	  \vfill
	
	\textbf{Gdańsk}
	
	\textbf{\hmwkDueDate}
	\end{center}
\end{titlepage}

\newpage
\setcounter{secnumdepth}{5}
\tableofcontents
\newpage

\section{Opis projektu}
\label{sec:Project}

Aplikacja projektowa jest przeznaczona do zarządzania ocenami filmów przez użytkowników. W ramach projektu zaimplementowano system bazujący na architekturze mikroserwisów, w szczególności wykorzystujący kontenery Docker oraz platformę Kubernetes do zarządzania klastrami kontenerów.

\subsection{Opis architektury}
\label{sec:Architecture}

Architektura aplikacji oparta jest na mikroserwisach w kontekście Kubernetes. Główne komponenty architektoniczne to:

\begin{itemize}
  \item \textbf{Backend}: Serwis obsługujący logikę biznesową aplikacji, zaimplementowany jako kontener Dockerowy uruchamiany w klastrze Kubernetes.
  \item \textbf{Frontend}: Aplikacja kliencka napisana w React, komunikująca się z backendem poprzez API HTTP.
  \item \textbf{Baza danych}: PostgreSQL wykorzystywany do przechowywania danych o filmach, użytkownikach i ocenach.
  \item \textbf{Keycloak}: Serwer do zarządzania tożsamościami i dostępem, zapewniający autoryzację i uwierzytelnianie użytkowników.
\end{itemize}

Każdy komponent aplikacji jest wdrażany jako osobny Deployment w Kubernetes, co zapewnia skalowalność i niezawodność systemu.

\subsection{Opis infrastruktury}
\label{sec:Infrastructure}

Aplikacja działa lokalnie przy użyciu narzędzia Minikube do lokalnego zarządzania klastrami Kubernetes. Minikube pozwala na uruchamianie jednoklastrowych środowisk Kubernetes na maszynie lokalnej.

Do uzyskania dostępu do aplikacji korzystano z port forwardingu, który przekierowuje ruch HTTP do odpowiednich usług w klastrze Kubernetes.

\newpage
\subsection{Opis komponentów aplikacji}
\label{sec:Components}

Komponenty aplikacji są dokładnie konfigurowane i zarządzane przy użyciu Kubernetes:

\begin{itemize}
  \item \textbf{Backend Deployment}: Kontener Dockerowy uruchamiany jako Deployment w Kubernetes, skalowany dynamicznie dzięki HorizontalPodAutoscaler.
  \item \textbf{Frontend Deployment}: Aplikacja React, umożliwiająca interakcję użytkowników z backendem poprzez API REST.
  \item \textbf{PostgreSQL Deployment}: Baza danych PostgreSQL jako usługa używana do przechowywania danych aplikacji.
  \item \textbf{Keycloak Deployment}: Serwer Keycloak odpowiedzialny za zarządzanie tożsamościami i autoryzację użytkowników.
\end{itemize}

\subsection{Konfiguracja i zarządzanie}
\label{sec:Configuration}

Konfiguracja aplikacji na poziomie klastra Kubernetes obejmuje ustawienia zasobów, takich jak CPU i pamięć, oraz zarządzanie dostępem do danych aplikacji przez ConfigMaps i Secrets.


\subsection{Skalowalność}
\label{sec:Scalability}

Aplikacja jest skalowalna dzięki użyciu HorizontalPodAutoscaler, który monitoruje obciążenie CPU i automatycznie dostosowuje liczbę replik backendu w zależności od potrzeb.

\subsection{Wymagania dotyczące zasobów}
\label{sec:ResourceRequirements}

Dla każdego komponentu aplikacji określono wymagania dotyczące zasobów, takie jak minimalne i maksymalne użycie CPU oraz pamięci, zapewniając odpowiednią wydajność i czas odpowiedzi dla użytkowników.

\subsection{Architektura sieciowa}
\label{sec:NetworkArchitecture}

Architektura sieciowa aplikacji obejmuje konfigurację w klastrze Kubernetes przy użyciu narzędzia Minikube do zarządzania środowiskiem. Dostęp do aplikacji uzyskiwany jest poprzez port forwarding, co umożliwia przekierowanie ruchu HTTP na odpowiednie porty usług w klastrze.

\newpage

% Bibliografia
\nocite{*} 

\bibliographystyle{plain}
\bibliography{references} 

\end{document}
